\documentclass{article}
\usepackage[margin=.9in]{geometry}
\usepackage{xcolor}
\usepackage{amsmath}
\usepackage{amssymb}
\usepackage{float}
\usepackage{listings}
\usepackage{natbib}
\usepackage{booktabs}
\setlength{\parindent}{0pt}
\setlength{\parskip}{\baselineskip}
\definecolor{mycolor}{rgb}{0.1, 0.1, 0.5}
\title{\textcolor{mycolor}{\textbf{{\huge Development and Implementation of a Low-Cost Electrochemistry Lab Kit for Educational Outreach: Literature Review}}}}
\author{Student: Christopher Hunt \\ Mentor: Dr. Kelsey Stoerzinger}
\date{}
\usepackage{graphicx} 
\usepackage{fancyhdr}

\begin{document}
\pagestyle{fancy}
\fancyhf{}
\rfoot{}
\lfoot{Christopher Hunt}
\lhead{Development and Implementation of a Low-Cost Electrochemistry Lab Kit for Educational Outreach: Literature Review}
\rhead{\thepage}
\maketitle
Electrochemical research holds immense potential to address challenges in energy sustainability and environmental conservation. This field encompasses work on renewable energy generation, energy storage, carbon capture, and environmental remediation, among others (citation). A cornerstone for propelling advancements in these areas is educating the next generation of engineers and scientists. However, the significant costs associated with essential instrumentation, coupled with a lack of educational resources, present considerable hurdles.

At the heart of electrochemical research lies the potentiostat, an instrument fundamental to a variety of experimental methods. Commercial potentiostats often retail for over a thousand dollars, a price point that restricts educational opportunities and limits the pursuit of electrochemical innovation (citation).

An increasing number of researchers have leveraged the rising accessibility of affordable microcontrollers, like the Arduino Uno, to develop low-cost potentiostats (citation). While these economical devices may not yet match the capabilities of their commercial counterparts, they serve as invaluable educational tools. Still a gap exists between the research and the standardization of design for academic use. Literature explored in this review has shown designs all meeting benchmark testing, however, these projects remain difficult to implement in a high school or undergraduate setting (citation). By introducing electrochemical techniques to students and resource-constrained communities, these low-cost potentiostats facilitate learning and stimulate innovation, despite financial limitations.

This review encompasses four potentiostat designs. A design by Gabriel N. Meloni from the University of Sao Paulo, refered here as The Meloni design. Another by Aaron A. Rowe, et al. from the University of Santa Barbara called the CheapStat. Another by Allison V. Cordova-Huaman, et al. called the PaqariStat. And another designed by Adrian Butterworth, et al. named the SimpleStat. These designs will be evaluated against specific criteria that align with our ambition to enhance education and outreach in Electrochemical Engineering. These criteria include: design complexity, functional efficacy, educational accessibility. 
 
\subsection*{Design Complexity}

\subsection*{Functional Efficacy}
\subsection*{Educational Accessibility}


\end{document}